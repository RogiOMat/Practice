\appendix{Фрагменты исходного кода программы}


Orderform.php
\begin{figure}[ht]
	\begin{lstlisting}[language=Php]
		// Функция получения компонентов и загрузка данных
		function getComponents($link, $table, $id_col, $name_col, $check_stock = false) {
			$data = [];
			$sql = $check_stock 
			? "SELECT $id_col, $name_col, stock FROM $table WHERE stock > 0"
			: "SELECT $id_col, $name_col FROM $table";
			
			$result = mysqli_query($link, $sql);
			if ($result && mysqli_num_rows($result) > 0) {
				while ($row = mysqli_fetch_assoc($result)) {
					$item = ['name' => $row[$name_col]];
					if ($check_stock) $item['stock'] = $row['stock'];
					$data[$row[$id_col]] = $item;
				}
			}
			return $data;
		}
		
		$customers = getComponents($link, 'customer', 'ctr_id', 'full_name');
		$masters = getComponents($link, 'master', 'mtr_id', 'full_name');
		
		$components = [
		'gpu' => getComponents($link, 'gpu', 'gpu_id', 'gpu_name', true),
		'mcase' => getComponents($link, 'mcase', 'cse_id', 'case_name', true),
		'motherboard' => getComponents($link, 'motherboard', 'mbd_id', 'motherboard_name', true),
		'powerunit' => getComponents($link, 'powerunit', 'psu_id', 'power_name', true),
		'processor' => getComponents($link, 'processor', 'cpu_id', 'unit_name', true),
		'ram' => getComponents($link, 'ram', 'ram_id', 'ram_name', true),
		'storage' => getComponents($link, 'storage', 'sdu_id', 'storage_name', true)
		];
	\end{lstlisting}
	\caption{Функция получения компонентов и загрузка данных}
	\label{fig:orderform_part1}
\end{figure}

\begin{figure}[ht]
	\begin{lstlisting}[language=Php]
		// Обработка POST-запроса и валидация данных
		$errors = [];
		$success_message = '';
		
		if ($_SERVER["REQUEST_METHOD"] == "POST") {
			// Валидация данных
			$assembly_price = $_POST["assembly_price"];
			if (strlen($assembly_price) > 6) {
				$errors['assembly_price'] = "Цена должна содержать не более 6 символов.";
			} elseif (!is_numeric($assembly_price) || $assembly_price < 0) {
				$errors['assembly_price'] = "Цена должна быть неотрицательным числом.";
			}
			
			$delivery_address = $_POST["delivery_address"];
			if (strlen($delivery_address) > 110) {
				$errors['delivery_address'] = "Адрес доставки не должен превышать 110 символов.";
			}
			
			$date_of_admission = $_POST["date_of_admission"];
			$date_of_delivery = $_POST["date_of_delivery"];
			
			if (strtotime($date_of_admission) > time()) {
				$errors['date_of_admission'] = "Дата оформления не может быть в будущем.";
			}
			
			if (strtotime($date_of_delivery) < strtotime($date_of_admission)) {
				$errors['date_of_delivery'] = "Дата доставки должна быть позже даты оформления.";
			}			
			$component_ids = [
			'gpu_id' => $_POST['gpu_id'],
			'cse_id' => $_POST['cse_id'],
			'mbd_id' => $_POST['mbd_id'],
			'psu_id' => $_POST['psu_id'],
			'cpu_id' => $_POST['cpu_id'],
			'ram_id' => $_POST['ram_id'],
			'sdu_id' => $_POST['sdu_id']
			];		
			.
			.
			.
			.
			.		   
	\end{lstlisting}
\caption{Обработка формы заказа и обновление базы данных}
\label{fig:orderform_part2}
\end{figure}

\begin{figure}[ht]
	\begin{lstlisting}[language=Php]
			if (empty($errors)) {
				mysqli_autocommit($link, false);				
				try {
					// Проверка остатков компонентов
					foreach ($component_ids as $field => $id) {
						$table = get_table_name($field);
						$stmt = $link->prepare("SELECT stock FROM $table WHERE ".get_id_column($table)." = ? FOR UPDATE");
						$stmt->bind_param("i", $id);
						$stmt->execute();
						$result = $stmt->get_result();						
						if (!$result || $result->num_rows === 0) {
							throw new Exception("Компонент $table не найден");
						}						
						$row = $result->fetch_assoc();
						if ($row['stock'] < 1) {
							throw new Exception("Недостаточно $table на складе");
						}}					
					// Добавление заказа
					$stmt = $link->prepare("INSERT INTO assembly (
					assembly_price, date_of_admission, date_of_delivery, delivery_address,
					ctr_id, mtr_id, gpu_id, cse_id, mbd_id, psu_id, cpu_id, ram_id, sdu_id
					) VALUES (?, ?, ?, ?, ?, ?, ?, ?, ?, ?, ?, ?, ?)");					
					$stmt->bind_param("dsssiiiiiiiii",
					$_POST['assembly_price'],
					$_POST['date_of_admission'],
					$_POST['date_of_delivery'],
					$_POST['delivery_address'],
					$_POST['ctr_id'],
					$_POST['mtr_id'],
					$_POST['gpu_id'],
					$_POST['cse_id'],
					$_POST['mbd_id'],
					$_POST['psu_id'],
					$_POST['cpu_id'],
					$_POST['ram_id'],
					$_POST['sdu_id']);	
					if (!$stmt->execute()) throw new Exception("Ошибка оформления: ".$stmt->error);
					// Обновление остатков
					foreach ($component_ids as $field => $id) {
						$table = get_table_name($field);
						$stmt = $link->prepare("UPDATE $table SET stock = stock - 1 WHERE ".get_id_column($table)." = ?");
						$stmt->bind_param("i", $id);
						if (!$stmt->execute()) throw new Exception("Ошибка обновления: ".$stmt->error);}
	\end{lstlisting}
\caption{Обработка формы заказа и обновление базы данных}
\label{fig:orderform_part3}
\end{figure}

\clearpage
\begin{center}
\textbf{Место для диска}
\end{center}
\fi
