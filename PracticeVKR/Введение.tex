\section*{ВВЕДЕНИЕ}
\addcontentsline{toc}{section}{ВВЕДЕНИЕ}

Создание персональных компьютеров является важной вехой в человеческой истории, с микрокомпьютерной революции прошло всего немногим более 50 лет, а персональный компьютер укрепился в повседневной жизни почти каждого человека. В 70-х годах прошлого века с появлением первых мини-компьютеров пришедших на замену огромным мейнфреймам, которые использовались только редкими крупными организациями, они были всё еще больше своих современных собратьев, но благодаря интегральным схемам имели уже гораздо более скромный размер. Со временем развитие компьютерных технологий привело к появлению персональных компьютеров.

Что такое персональный компьютер и в чем его преимущества? Это относительно дешевое и компактное устройство, благодаря которому пользователь способен решать огромное множество задач. Современные компьютеры объединяют весь мир с помощью сети интернет, позволяют людям за доли секунд доставлять письма, изображения, данные на другой край планеты, искать почти любую информацию в этой-же самой сети. Помимо взаимодействия с самой сетью ПК позволяет пользователю проводить сложные вычисления, расчеты, моделировать, вести отчетность, работать с большими объемами данных в удобном формате.

Основная задача разработки в том, чтобы увеличить количество посетителей и заказчиков, предоставить удобную систему как для работы сотрудников, так и для посетителей. Если достаточно заинтересовать пользователя, тот с большим шансом решит воспользоваться услугами нашего сервис-центра.

\emph{Цель настоящей работы} – разработка веб-приложения сервис-центра для привлечения новых покупателей, увеличения заказов, рекламы продукции и услуг. Для достижения поставленной цели необходимо решить \emph{следующие задачи:}
\begin{itemize}
\item провести анализ предметной области;
\item разработать концептуальную модель веб-приложения;
\item спроектировать веб-приложение;
\item реализовать веб-приложение.
\end{itemize}

\emph{Структура и объем работы.} Отчет состоит из введения, 4 разделов основной части, заключения, списка использованных источников, 2 приложений. Текст выпускной квалификационной работы равен \formbytotal{lastpage}{страниц}{е}{ам}{ам}.

\emph{Во введении} сформулирована цель работы, поставлены задачи разработки, описана структура работы, приведено краткое содержание каждого из разделов.

\emph{В первом разделе} на стадии описания технической характеристики предметной области приводится сбор информации о деятельности сервис-центра, для которой осуществляется разработка сайта.

\emph{Во втором разделе} на стадии технического задания приводятся требования к разрабатываемому сайту.

\emph{В третьем разделе} на стадии технического проектирования представлены проектные решения для веб-приложения.

\emph{В четвертом разделе} приводится список классов и их методов, использованных при разработке сайта, производится тестирование разработанного сайта.

В заключении излагаются основные результаты работы, полученные в ходе разработки.

В приложении А представлен графический материал.
В приложении Б представлены фрагменты исходного кода. 
